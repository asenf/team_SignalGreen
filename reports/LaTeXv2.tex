% \usepackage{ }
\documentclass[11pt]{article}

%\usepackage[a4paper,margin=3cm]{geometry}

\usepackage{geometry}
\geometry{a4paper,margin=3cm}

\usepackage{graphicx}



\begin{document}




\title{Signal Green Initial Report}

\author{Waqar Aziz, James Kerr, Adeela Saalim, Andrea Senf, Yoann Strigini}

\maketitle 

\section*{}

\subsection*{AIMS FOR PROJECT}



Our aim is to create a simulation engine where we can test various traffic management 
strategies that will teach us new skills and receive full marks. 
\\ \\
This means we will demonstrate our ability to use GitHub and other common engineering 
tools to create a vehicle simulation. It includes working together as a team and learning 
from each other. We will experience developing a software that 
is bigger than any of us could have created by ourselves in the given time. 


\subsection*{STRATEGY}

We are coding in java using Eclipse editor. We are using java because everyone 
on the team is experienced in using it and java is commonly used for applications in industry. 
Its ease of use compared to C++ will help us expedite programming progress.
\\ \\
We decided to use the RePast Simphony framework. It is flexible, allows 
module replacement if we desire, and is more recent than RePastJ. We are using the Eclipse 
editor because it comes set up with Simphony and it is commonly used.
\\ \\
We further considered and decided against using NetLogo, coding the ABM software ourselves, and RePastJ. 
We decided coding an entire ABM software ourselves would not leave sufficient time to 
complete other project requirements. Although NetLogo has many preexisting models, it is 
also rather restrictive; there was also concern about the final appearance of the project.
We decided against RePastJ because it is an older version than Simphony.
\\ \\
We are communicating and documenting via WhatsApp, asana and Instagantt, and thus 
far we have held weekly team meetings in person at KCL to discuss research and divide tasks.
\\ \\
Every week team members report on work done via asana to ensure progress is continually 
being made. Asana is connected to Instagantt, where we can see the corresponding Gantt
chart of our progress.
\\ \\
The week of 23 February our program will (at minimum) run a basic simulation, and the 
group will assess our progress and set final goals and specifications for our finished 
project.

 
\includegraphics{Gantt31Jansmall}



\subsection*{CURRENT PROGRESS}

Our team is functioning on GitHub and we are currently coding the architecture and 
environment. The basics of choosing and learning new enviornments is complete, and we
are now focused on making progress with the program itself and continuing documentation.
 

\subsection*{TEAM WORK}

We communicate in person and via our selected communication tools.  Team members 
volunteer or recommend peers for tasks that play to that team member's strength or where 
that person would like to learn new skills.


\subsection*{ROLES}

James and Yoann are focusing on the program architecture and vehicle classes. Waqar and 
Adeela are focusing on creating the road map components. Andrea is coordinator for GitHub 
purposes and is creating documentation and reporting. All team members will participate 
in testing and analysis.


\subsection*{PEER ASSESSMENT}

We expect everyone to carry roughly equal amounts of work and so we plan to divide team 
points equally between us. If we have concerns that this is not happening, we will 
discuss it so there will not be surprises in our assessment of each other at the end of term.


\subsection*{CONFLICTS}

Concerns or issues between team members should be discussed between the members within 
a reasonable time of noticing the problem. This is to prevent surprises about members' 
evaluation of each other at the end of the project, and to minimize friction between 
group members.
\\ \\
We anticipate the load should be shared fairly equally between members and 
this is considered when taking tasks during team meetings. If a member feels 
they are doing more work than others, this should be communicated to the group so action 
can be taken while there is time to remedy the issue.



\end{document}
