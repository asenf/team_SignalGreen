
\documentclass[11pt]{article}

\usepackage{geometry}
\geometry{a4paper,margin=3cm}

\usepackage{graphicx}
\usepackage{enumitem}
\usepackage[T1]{fontenc}
\DeclareGraphicsExtensions{.pdf,.png,.jpg,.jpeg}



%  SPECIAL CHARACTERS
%  \textquoteright      - apostrophe (‘)
%  \textquotedblleft     - quote(“)
%  {\_}                      - underscore (_)
%   \%                             - percent (%)

% \begin{itemize}\itemsep0pt
%  \item The first item
%  \item The second item
%  \item The third etc \ldots
% \end{itemize}



\begin{document}

\title{Signal Green Final Report}

\author{Waqar Aziz, James Kerr, Adeela Saalim, Andrea Senf, Yoann Strigini}

\maketitle 


\section{INTRODUCTION}


This work was first and foremost an opportunity for working in a group. Employers want engineers who are not only skilled in their craft, but also persons who are able to collaborate, work well as part of a team, and invest in project aims and those working to achieve them. Our team was able to develop a simulation that involved each member in the process. In spite of differing abilities and communication styles, each person was able to contribute in important ways.

The software engineering part of this project focused on creating a traffic simulation software where we can test various traffic management strategies. We wanted to create a simulation that would be visually interesting with user modifiable variables, and that would have potential for further expansion and use.

We were able to create a simulation that is interesting, flexible, and allows different layouts and road policies to be tested and visually represented. Any geographic information system (GIS) map shapefile representing road networks can be loaded into our simulator and could easily be extended to take into account any other relevant attributes.

Our team further gained a more practical understanding of agent based modelling (ABM) through using Repast Simphony, and gained experience with version control through using GitHub.

\section{BACKGROUND}

\subsection{TRAFFIC FLOW FORECASTS}

Analysis of complex urban networks and traffic flow models is the groundwork for reliable traffic flow forecasts, which are widely used to avoid traffic congestion and maximise road network capability in metropolitan areas.[11k] Predictions of traffic growth published by the UK Department of Transport in 2013 show that despite a slowdown in the past decade mainly due to economic recession and high oil prices, traffic on all roads in England are expected to grow by about 45\% by 2040 [10j]. 


\begin{figure}[h]
\begin{center}
\includegraphics[scale=0.7]{england_traffic}
\caption{England traffic, actual and predicted}
\end{center}
\end{figure}


Consequently, causes and effects of traffic congestion have been studied extensively in order to face the increasing road network saturation. Several factors that have significant impact on traffic flow have been identified, including the following: 
\begin{itemize}\itemsep0pt
\item Road/Traffic policy approaches
\item Timing/Cost of journeys
\item Types of vehicles and speed
\item Junctions and bottlenecks
\item Lane splitting and joining
\item High occupancy vehicle lanes
\item Driver\textquoteright s behaviour
\item Rush hours
\item Occasional factors such as car accidents, road works or bad weather
\end{itemize}

The model we implemented has been designed with acceptable trade-off between accuracy and computational complexity. We have therefore selected the following set of urban traffic factors:
\begin{itemize}\itemsep0pt
\item Different types of vehicles, e.g. trucks are slower than cars and are less likely to overtake
\item Driver\textquoteright s behaviour, e.g. reckless drivers do not stop at amber signal
\item Journey of vehicles is based on best route according to roads\textquoteright length and speed limit
\item Roads can have multiple lanes, fast vehicles try to overtake slow ones
\item Different traffic management policies, including traffic lights and/or give way signs
\end{itemize}

The ultimate goal of traffic simulation is to create space for people to move along their journeys in a safe manner with consistent progress. 

Data extracted from simulations will be evaluated at the end of this report, in order to analyse different traffic policies highlighting their pros and cons.

Our team began by researching various traffic simulation models and code available online. Some of the more significant included. 

[1a] gave us a picture of what an interesting simulation and its code might look like. 
[2b] offered a good description of what coding with cellular automata involves for those on the team unfamiliar with the concept. In the early stages of the project we thought coding using cellular automata might be an approach we would utilize; this was later decided against in favour of ABM.
[3c] show one way to conceptualize moving cars along a road from origin to destination. 
[4d] Detailed example of an agent-based traffic simulation; referenced for approach and features often considered when creating an agent based model. Well written and fairly comprehensive.
[5e] argues that \textquotedblleft only agent-based models can explicitly incorporate the complexity arising from individual behaviors and interactions that exist in the real-world. \textquotedblright [5e 1458] Shows the wide uses of Repast in various fields [5e 1457 Table 1]. The article notes that agent modelling using desktop environments like Repast Simphony is a good way to explore the potential of ABM in a brief space of time and with minimal training investment [5e 1464] and this is a benefit our team hoped to gain through this project.
[6f] provided many ideas regarding behaviour rules that was useful, as was the brief but descriptive discussion of modelling.
[7g] The entire team went through a very good tutorial to learn Simphony basics. This tutorial was an excellent introduction to Simphony and had clear parallels to our simulation. 
[8h] used as reference for coding lane changing and gap acceptance logic.

Selected Bibliography [do not renumber, these are throughout the paper]:
%[1a] https://optalk2011.wordpress.com/traffic-simulator1/
%[2b] http://natureofcode.com/book/chapter-7-cellular-automata/
%[3c] http://crimesim.blogspot.it/2008/05/using-Repast-to-move-agents-along-road.html
%[4d] http://www.myhomezone.co.uk/project/Report.htm
%[5e] http://www.informs-sim.org/wsc11papers/130.pdf
%[6f] http://www.umc.edu.dz/vf/images/misc/session4A/34-4A-paper2-Benhamza-SERIDI%20Hamid.pdf
%[7g] http://repast.sourceforge.net/docs/RepastJavaGettingStarted.pdf
%[8h] https://dspace.mit.edu/bitstream/handle/1721.1/34607/71301166.pdf?sequence=1
%[9i] https://github.com/geotools/geotools/blob/master/modules/unsupported/process-raster/src/main/java/org/geotools/process/raster/GridConvergenceAngleCalc.java
%[10j] https://www.gov.uk/government/publications/road-transport-forecasts-2013
%[11k] https://www.gov.uk/government/uploads/system/uploads/attachment_data/file/230433/an-introduction-to-dft-congestion-statistics.pdf
% [13n] https://dspace.mit.edu/bitstream/handle/1721.1/34607/71301166.pdf


\section{REQUIREMENTS AND DESIGN}


Our traffic simulator is an agent-based, non-deterministic, discrete-time simulator for microscopic traffic modelling which allows different traffic policies on user-defined Geographic Information System (GIS) maps to be tested. 

An Agent-Based modelling (ABM) simulation seemed most appropriate for a traffic simulation[5e, see 1465]. ABM uses entities (agents) to embed certain behaviours and they exist in a certain environment. Agents are autonomous, have objectives and goals, and can interact with other agents. ABM is particularly suitable for traffic simulation because agents, such as vehicles and traffic lights, can adapt to the evolving environment and accommodate changes, e.g. a vehicle needs to accelerate when a traffic light turns green.

The system is non-deterministic, with future states of traffic flow depending on unpredictable events, just as real-life traffic phenomena always have a degree of unpredictability. Randomness is added by the use of a random seed which changes during each run: with same input, output is always different. 

A discrete time model, given its iterative nature, is convenient for traffic simulations. Every tick the current state of the agents is evaluated, algorithms are recomputed, and the overall state of the system is updated. For example, traffic lights change light every n ticks, and each vehicle\textquoteright s velocity is updated accordingly. Our simulation updates state every tick of the clock.

Microscopic traffic models typically simulate the behaviour of single vehicles using microscopic properties such as the vehicle\textquoteright s position and velocity. Our simulation uses the following common decision models[13n]: Car-following model, Lane-selection model, and Gap-acceptance model.

Lastly, GIS integration allows real maps to be loaded into the system. GIS is the de-facto standard for most professional traffic simulators, and governments regularly publish GIS maps about road networks, which are freely available for download. Virtually any GIS map can be loaded into our system.

\subsection{Milestones}

For the purposes of planning, our team set Milestones to measure what we wanted to accomplish. This is our plan as it stood at the beginning of February: 
\\
Milestone 1:
\begin{itemize}\itemsep0pt
\item deadline: 23 February
\item code compiles and runs simple simulation
\item vehicles run on a map
\item variable number of vehicles
\item vehicles make decisions to reach a goal
\end{itemize}
\\
ztone 2:
\begin{itemize}\itemsep0pt
\item deadline: 8 March
\item implement basic vehicle type: car
\item traffic flows bi-directionally
\item junctions
\item signals, give way, roundabouts
\item multiple maps in GIS standard
\item cars have different behaviours: timid, aggressive, patient
\item multiple lanes on some roads
\end{itemize}
\\
Milestone 3:
\begin{itemize}\itemsep0pt
\item deadline: 16 March
\item  vehicles appear visually different by some criteria (type/behaviour, speed, congestion)
\item  speed limits on roads
\item  implement vehicle types: lorry, emergency, motorcycles
\item  vehicles exhibit passing behaviours
\end{itemize}
\\
Possible Traffic Policy Implementations:
\begin{itemize}\itemsep0pt
\item all light junctions
\item  all roundabout/give way junctions
\item  mix lights/roundabout junctions
\item  variable speed limits
\item  same junction policy with varying levels of congestion
\end{itemize}
\\

\subsection{Language and Framework}

The team began by choosing a programming language for development. We quickly settled on Java as our language, as everyone on the team was familiar with it and Java is commonly used in industry. Java is fast enough to simulate hundreds of agents, is entirely object oriented which makes it ideal for an agent paradigm, and currently the most popular Agent Based Modelling framework is in Java. C++ was considered, but the team preferred Java for its ease of coding. A scripting language like python would have enabled fast development, but team members were unfamiliar with it, it is computationally slow, and so not a good choice for modelling with many objects.

For a framework we considered several options. We could code an entire agent based traffic simulation ourselves; this was not selected due to the enormous coding task this would involve, and we were concerned we would not meet all the requirements for the assignment in the time allocated. We considered NetLogo, but decided it did too much work for us; we would do almost no coding ourselves, and very little logic to create. We wanted to show our own ability to code and create our own logic.


\subsection{Repast Simphony}

We settled on Repast Simphony, a general application programming interface (API) providing us tools, libraries, and plug-ins to create a simulation while allowing us to code and implement appropriate logic for the desired simulation. Repast comes with a BSD software license and is widely used in many fields of academia. [5e, table 1, 1457] Repast Simphony provided the following benefits that made it a good fit for our group project:

\begin{itemize}\itemsep0pt
\item As a framework for Java, it is fully object oriented and everything is created as a Plain Old Java Object. It includes Java libraries and .jar utilities, and all objects are rewritable.
\\
\item It provides an ABM environment with a practical skeleton of agents and their contexts, and includes features such as behaviour activation, a discrete event scheduler and Watcher component, and space and grid management.
\\
\item It provides a flexible plug-in framework, allowing the developers to use, modify, or write plug-ins. It provides the ability to integrate advanced elements into our code such as importing GIS maps, multi-threading, XML configuration files, and other advanced features.
\\
\item It includes charting and data collection functions, so we can focus on deciding on traffic policies and changing these parameters rather than spending time finding ways to display and record the results. The probe function is particularly useful for ABM testing.
\\
\item It provides an attractive display graphical user interface (GUI) we can modify to suit our needs and make a more interesting simulation, including user defined options at runtime and 2D and 3D visualisation and GIS plugins. One can create new or rewrite existing plug-ins if one wishes.
\end {itemize}

The benefits of Simphony come at the flipside cost of there being more opportunities for our team to display skill in modelling concepts and logic rather than a mastery of Java programming skill. When our team chose ABM as our design platform this meant behaviour and goals became more central than designing passive functions [4d], and we trust our ability to create an interesting system shows in our project and does credit to our ability. The team spent some time learning the framework, but this was considered a good investment for the benefits.

\\ 

\section{IMPLEMENTATION}

\subsection{Simphony and Our Code}

The integrated development environment (IDE) for Simphony is a preconfigured Eclipse IDE. Eclipse Simphony varies from the normal Eclipse IDE in that it contains custom views, run configurations, and it adds three .jar files to the buildpath: JOGL (Java bindings for openGL), Java3D, and JAI (Java Advanced Imaging). These .jar files are used to create graphical representation of objects. Simphony also requires a ContextBuilder to initialize and drive the simulation, replacing the Java Main() run method; in our code class SignalGreenBuilder.java implements the ContextBuilder for the program.

While Simphony provided a very useful API, our team determined how each agent and behaviour should be modelled and the logic used to describe each object. For example, Simphony had no effect on how we modelled traffic signals. There is no agent superclass, no inheritance interfaces, no exposed code provided. The actual creation of agents and the method of that creation is the work of the programmers. 

Simphony provided nice helper features and many tools to increase the speed at which we could code. Our code uses its background foundation required for ABM, including multi-threading, the Clock/Watcher, having objects automatically drawn on the screen, and agents being aware of their environment. While our team wanted to especially take advantage of Simphony\textquoteright s visualisation benefits, our core requirements are based around the logic of the agents, not their appearance.


\subsection{Agents and Environment}


The architecture for our simulation engine is a hierarchy of Java classes, where each class is either an agent or an environment. Agents are objects that exhibit behaviours and can either be of fixed geography (eg: junction, road) or they can move on the GIS projection (eg: vehicles and their subclasses). Agent behaviour can be embedded through the agents reacting to the environment themselves, by using the watcher component which listens for events and triggers actions, or by globally scheduling events to happen.

An environment forms a container for agents and includes the Road Network Topology (the network), and GIS Geography (the geography). These are loaded from standard GIS shapefiles (.shp extension). The environment is coded in continuous space with (X,Y) coordinates on a directed graph network, with the Network being the topological relationship between each road segment. At any time each agent knows its position on the Geography; vehicle agents can ask the Network for their route and then follow this course along each road segment. Continuous space also allows the user interface to display agents in detail during the simulation.

Repast offers a ContextBuilder component containing a framework for creating agents and environments. The team used the framework to model entities and features including (but not limited to): vehicles, roads, junctions, road network topology, and GIS geography. We also coded all behaviour logic (direction, volition, velocity, etc) for the agents.

We used Styled Layer Descriptor (SDL) files to create the presentation; SDL is an extension of OpenGIS which is open source and provides the benefit of XML format.


\subsubsection{The SignalGreenBuilder Class}

Class SignalGreenBuilder.java builds the road network and the GIS geography for our simulation and initialises the agents. It is the only ContextBuilder class for the program. At the highest level of abstraction, the program Context contains a graph (Road Network) of vertices (Junctions) connected by paths (edges). The Road Network is a directed graph representing the road topology, and the GIS geography is used to display the simulation. 

Repast has built-in GIS integration, so reading in spatial data is done by iterating over each Feature object of the ShapefileDataStore file (shapefile). Every feature of type \textquotedblleft MultiLaneString\textquotedblright represents a road object with its spatial location and other attributes. Creating a road network is achieved by linking edges together, and the road network topology is built with pairs of connecting junctions.

Any GIS map shapefile (.shp) representing road networks can be loaded into our simulator. The simulator uses some GIS attributes for modelling vehicle\textquoteright s behaviour (ex. road type, max speed, etc.) and could easily be extended to take into account any other relevant attribute.
GIS maps were simplified by removing the road coordinates between the Junctions to make it more efficient; creating road agents with only the endpoint coordinates specified prevents very complex urban street maps from becoming too computationally expensive.[see figure XXX] 
\\

\begin{figure}[h]
\begin{center}
\includegraphics[scale=0.4]{creating_road_segments_from_shapefile}
\caption{creating road segments from shapefile}
\end{center}
\end{figure}

\\
Each Road agent is added to both the Context and the GIS geography, along with two Lane agents for each side of the Road (used only for displaying vehicle graphics on the UI). A Repast edge is then added to the road network, and the Repast edges are then linked together to create the road network topology. This is done by keeping a cache of all previously created Junctions, so every time a new Road agent is created the cache is checked to see if there already exists Junctions for its end point coordinates. If there is, then we know all roads linked to those Junctions are also linked to the Road agent being created, and we do not need to create any more new Junctions. [3c for more detail] 
\\

\begin{figure}[h]
\begin{center}
\includegraphics[scale=0.4]{creating_new_roads}
\caption{How to create new roads}
\end{center}
\end{figure}

\\
As an example, consider figure [concrete example].

Before R3 is added:
Cache of Junctions = (J1, J2, J3)
Road network = (R1, R2)
Road network topology = (<J1, J2>, <J2, J3>)

After R3 is added:
Cache of Junctions = (J1, J2, J3, J4)
Road network = (R1, R2, R3)
Road network topology = (<J1, J2>, <J2, J3>, <J2, J4>)

Once the whole shapefile has been read, the GIS geography is displayed on the UI, as shown on [figure Display Manhattan]. In this example we have loaded a map of the main avenues of New York City and clicked on a Road agent to show its attributes: name = Madison Av, speedLimit = 65 Km/h, length = 4,805 meters. Virtually any urban street map can be loaded into Signal Green.
\\

\begin{figure}[h]
\begin{center}
\includegraphics[scale=0.4]{GIS_display_manhattan_loaded}
\caption{Manhattan!!}
\end{center}
\end{figure}
\\

\subsection{The Vehicle Class}


Class Vehicle.java is a very important class which initializes vehicles, assigns vehicles their individual behaviour, type, source, and interactions, and contains the generic vehicle constructor. Classes CarVehicle.java and TruckVehicle.java extend the Vehicle.java class and override the constructor to create three new types of vehicles: slow cars, fast cars, and trucks. Each type has its own adjusted maxVelocity and visual icon. 

Method initVehicle() creates a type of vehicle agent and assigns it a random source and destination junction that includes the vehicle\textquoteright s route to that destination. For performance reasons we did not find it feasible to calculate the best path at each vertex; instead the entire path is chosen at initialisation using the highly optimised shortest path algorithm provided by Repast Simphony. The path is shortened as each junction is reached on the way to the destination. When the destination is reached, new random values are chosen and the process repeats.

Method computeDisplacement() is used to determine exactly how far in metres the vehicle should travel based on the vehicle\textquoteright s attributes. 


\subsubsection{Method Step()}


A vehicle\textquoteright s vision is the distance around which a vehicle is aware of its environment and is calculated based on constant values DIST{\_}VEHICLES and DIST{\_}VEHICLES{\_}STOPPED. Method step() determines individual vehicle behaviours, including interaction with junctions and other vehicles, increasing/decreasing speed, changing lanes, and moving the vehicle along the map.

The vehicle checks the speed and distance of any vehicles in its path, slowing if needed to an optimal velocity. If a vehicle observes a slower moving vehicle in its path, it makes a decision whether or not to change lanes. The vehicle also checks for approaching junctions, decelerating to a stopped position before the light if a red light is in view. If no other vehicles or junctions are in its vision it will accelerate to its maxVelocity. 
\\

\begin{figure}[h]
\begin{center}
\includegraphics[scale=0.45]{step_method_seq_diagram}
\caption{Sequence diagram}
\end{center}
\end{figure}

\\
The method step() works as follows. (See Figure .) Assume the current vehicle V is heading towards a junction J[i]. J[i] is the next node on its route, J[1..n], where n is the number of nodes, J[1] and J[n] are V\textquoteright s origin and destination. 

V needs to first decide which lane it wants to go to (lane selection decision model, step 1.2) and check that the road is clear (gap acceptance decision model, step 2). 

In steps 3 and 4, V tries to either accelerate or slow down, depending on the outcome of the previous steps. 

Next, if V is close enough to J[i], traffic management policies are evaluated (step 8): for example, if J[i] uses traffic jams, then V asks J[i] if it has to wait (Light.RED signal detected, step 8.1.1). Finally, displacement is executed, meaning V\textquoteright s position is updated both in the road network topology and in the GIS projection (step 9).

V checks what is the J[i] on the route J[n]. It knows which vehicles are approaching because it holds a priority queue of incoming Vehicles for each road segment, where weights are the distance from V to J[i]. This is an efficient solution as insertion is done in O(log n), while removal is done in constant time.
\\

\begin{figure}[h]
\begin{center}
\includegraphics[scale=0.4]{overtaking}
\caption{Faster vehicle overtakes slower truck and merges back on the outer lane.}
\end{center}
\end{figure}

\\
From Repast\textquoteright s point of view, Step() method is executed every iteration of the simulation, using the Java annotation: @ScheduledMethod(start = 1, interval = 1) meaning that it starts from tick one, and gets executed every tick.


\subsubsection{Simulation Calibration}

Manual testing of SG was fundamental for proper calibration of simulation parameters.
Improper calibration leads to invalid output, thus special care must be taken when adjusting values to make the simulation visually realistic. The Constant class holds all arbitrary calibration parameters.

Simulation parameters and relevant measures are summarised as follows:
\begin{itemize}
\item 1 simulation tick = 1 second
\item GIS projection distances reflect real distances, and are measured in meters
\item Minimum distance between vehicles to avoid collisions = 1.8 meters
\item Arbitrary acceleration factor of vehicles = 1.6
\item Real vehicle displacement in meters = displacement * 70
\item Displacement equation:  x = v0 * t + 1/2 a * t * t, where x is the displacement, v0 the initial velocity, a the acceleration, t the time – which is arbitrarily set to 4.
\item Truck maximum speed = 80 Km/h
\end{itemize}


\subsubsection{Moving on GIS Geography}

The vehicle.java class determines the behaviour of vehicles so they follow roads. This required special coding, as GIS images point true north and use azimuth coordinates for its 3D mapping, while the UI points to grid north and uses simple (X,Y) coordinates for the grid. 

Azimuth coordinates allow one to measure distances and plot locations in a spherical coordinate system, and our code utilized this form of measurement in moving the vehicles along the maps. We referenced [9i] to see how one could compute the correct angle for displaying graphics on the UI. 

This further required the use of kinematics equations for plotting the direction of vehicles in several methods. Kinematics is the study of objects moving through space. Because we used GIS maps, all calculations along the map involve real distances.
\\

\begin{figure}[h]
\begin{center}
\includegraphics[scale=0.3]{moving_vehicles}
\caption{Vehicle V moving on its planned route.}
\end{center}
\end{figure}
\\

For example, consider Figure .  The following are the main steps:
\begin{enumerate}
\item V is heading towards J3 so it needs to compute the distance to it. 
\item V computes its current displacement. Displacement is computed using a standard kinematics equation which takes into account velocity and acceleration.
\item There can now be two cases:
\begin{itemize}
\item If V cannot make it all the way to J3, V just moves towards it by the whole displacement in meters.
\item If V is close enough to J3, meaning that current displacement > distance to next junction, V first moves to J3, updates its route (i.e. a pointer to the next junction) and then performs the remaining displacement towards the new next junction.
\end{itemize}
\end{enumeration}



\subsubsection{Junction Classes}

Class Junction.java creates junction objects that are nodes on the road network. These objects contain references to road segments (including adjacent junctions and roads) and queues (containing vehicles) associated with it. It is used throughout the program to determine where vehicles are, and so is fundamental to vehicle behaviour.

Class TrafficLight.java extends Junction.java and schedules traffic light management. It has its own step method to iterate through lights, using class Light.java which encapsulates and toggles between the three light states.

In order to implement traffic management policies, some junctions need to embed special behaviour. Signal Green supports traffic light logic or give way signs, however the model can easily be expanded by extending the Junction class. We assume that junctions with more than two road segments need some traffic management logic, so they will be either instances of TrafficLight or GiveWaySign.

Traffic lights have one instance of Light for each road adjacent to that junction, each light has a signal attribute which can be either RED, GREEN or AMBER. Lights change their state every 25 ticks, at any time only one Light for that junction can be in GREEN state. Vehicles that are exiting a road segment check if the Junction is an instance of TrafficLight, in which case they inspect the signal to know if they must stop.

\begin{figure}[b]
\begin{center}
\includegraphics[scale=0.4]{traffic_light_junction}
\caption{Junction of type TrafficLight.}
\end{center}
\end{figure}


In proximity of give way intersections, on the other hand, vehicles check for each road segment which vehicle is closest to that intersection, to know who has precedence. (See Figure .)

\begin{figure}[h]
\begin{center}
\includegraphics[scale=0.2]{give_way_junction}
\caption{Junction of type GiveWaySign.}
\end{center}
\end{figure}




\subsection{Other Work}

The only class which is not our team\textquoteright s work is class PriorityBlockingDequeue.java. This class is copyrighted by Aviad Ben Dov, and it implements a rather complicated data structure holding vehicles in a queue at junctions. We used his work because the code was complex enough that it would take a great deal of time to write ourselves, and we thought it prudent to use his code with attribution instead of risking not completing the junction code satisfactorily in time or relying too heavily on his work.



\section{TESTING}


Testing will be done to confirm that our system conforms to functional requirements and expected behaviours. Exhaustive testing of the system is naturally not realistically possible in the time given, but we will perform basic unit, functional, error, cross-platform, and system testing. 

ABM testing will be done by running the simulation, observing the agents\textquoteright  behaviour and comparing it with expected behaviour. We will be test our code at three different levels of functionality:

\begin{enumerate}\itemsep0pt
\item Agent behaviour: Verified by checking the changing environment variables around agents. This tests functional requirements of the system as well.
\item Runtime Parameters: Elements and data which load at runtime will be tested using valid and invalid inputs and checking for correct results. 
\item Overall system behaviour: Checks that all agents of the system produce expected results for a given scenario.
\end{enumerate}

This will include testing of agents interaction with other components of environment (roads, traffic lights) when programme will be executed at run time.

Automation testing will not be possible due to the nature of agent-based modelling simulations. Agents are non-deterministic and there is no set predictability of where objects will be at a given time. There is not a set result that can be expected for each vehicle, so automated testing for agents is not generally possible. 


\section{TEAM WORK}

Our approach to the project was to set down ideas, goals, and tools in the first two weeks, then jump into the coding and get as far along as we could until the initial report came due. This worked well in many ways. Important decisions about implementation were decided very quickly so coding was able to start in January. The milestones were itemized early in the project, with Milestone 1 set early and items assigned to Milestones 2 and 3 differentiated closer to the deadline of the Initial Report.


\subsection{Roles and Task Division}


Yoann Strigini: Design Architect, Developer One 
Design SignalGreen model, integration with GIS, develop Vehicle class, presentation layer, documentation (architecture and technical details)
Adeela Saalim: Testing Specialist
Develop test cases, run test cases, document testing
James Kerr: Documentation Specialist Two, LaTeX Specialist
Writing technical documentation for report
Waqar Aziz: Developer Two, Head Policy Comparison and Reporting 
Develop reporting functionality; primary traffic policy testing and reporting
Andrea Senf: Team Coordinator, Tech Doc Specialist
Coordinate team activities, Scrum Master, Lead documentation and reporting



\subsection{Tools}

\begin{itemize}\itemsep0pt
\item GitHub. GitHub was used for creating the code, providing version control and easy sharing of code. Initial and final reports were also kept there. Andrea held the main code repository, and the team branched off that code.
\item Communication. For communication outside coding, the most used tool by far was WhatsApp; this was practical for team chat, answering basic questions, and coordinating meetings. Longer reports for the group were written on asana or sent by email. Asana was used consistently by team members and was helpful for the documentation of the final report and for automatically generated Gantt charts through Instagantt.
\item Meetings. When one or more group members believed a whole group meeting was needed to further the project, this was communicated on WhatsApp and a time and place of meeting was arranged using input from all members. These meetings were in person. Other times two or three members met together using Skype or in person as needed to work on parts of the project pertaining only to them.
\end{itemize}


\subsection{Development and Reporting}

The team set out to utilize an Agile method of development with SCRUM reports once a week, either in person (if we had a meeting) or written if we did not gather. Group SCRUM did not pan out exactly as anticipated, as early meetings had absences which made SCRUM impractical as we waited for other members arrived (which they occasionally did not). We then decided to transition completely to written reports. 

Approximately every week the group coordinator would remind the team to post WhatsApp/asana updates so everyone was aware of what the others were doing and to help the team to continually progress. Team members were to request help as needed and there was always a visible \textquoteright next task \textquoteright\ waiting when one\textquoteright s current task had been completed, so while SCRUM was not overtly performed the goals behind the method did remain.


\subsection{Team Challenges}

Our team faced several challenges in working together and with our chosen tools. [Note: all members are here referred to as \textquoteright he\textquoteright .] 

\begin{itemize}\itemsep0pt
\item Our team struggled to coordinate regarding GitHub the first part of the project, as several members focused on coding on their own machines rather than branching from GitHub in the weeks prior to the initial report date. Initial code was uploaded to GitHub the day before the initial report was due. The members leading in coding later created a full GitHub project for everyone to access. 
\\
\item A member missed two meetings where decisions were made, so his preferences were not implemented. As he had strong feelings about these decisions it was difficult for the person to accept and caused some friction.
\\
\item A member found he did not have the ability to run Simphony and so did not have a way to code his part of the project; the group was notified of this after the initial report presentation. He found a way to partially resolve this so he could code from the last week of February, although still without full Eclipse functionality. 
\\
\item A member had difficulty prioritizing his code at the end, putting off resolving a key issue for several days. This resulted in our final Milestone deadline being pushed back almost a week and added stress to other team members.
\\
\item After difficulty getting team members to follow through with attending agreed meetings, we voted to modify our practice so that absence to agreed full group meetings would result in a point for that person being deducted.
\end{itemize}
\\


\section{EVALUATION}


\subsection{Group Project}

University group projects can be tricky to make work well; professional teams generally have the benefits of prior working relationships and clearly understood authority and accountability structures. SignalGreen team members were generally unfamiliar with each other before forming the group; four of our team formed based on where we were sitting in the introductory lecture, and the fifth was a person who we heard was looking for a group. 

Distributing points is generally difficult to manage well, as giving too much power to democratic vote can be seen as unfair and hurtful by some members, and not allowing enough flexibility can result in members not feeling impressed to participate or work in a timely fashion. 

In general, as masters students we are committed to doing good work, as our work for the project is part of our training for our careers. On this basis our team decided to share points evenly, and points lost due to agreed criteria would be distributed by vote. Some members did more work than our points can reflect, and we trust their hard work will pay off in personal dividends in the future.

Ultimately, the goal of the team was for everyone to get through the module with a pass (preferably much better). To this end, we are not group critiquing each other, nor are we calculating what we ourselves did for comparison to others. As we believe our team has been successful in the project we feel it would be better to end with a feeling of satisfaction for having completed a quality project.


\subsection{Milestone Results}

All objectives set for Milestone 1 were completed in its entirety by the deadline set by the team. At that time we extended the deadlines of the Milestones to have all coding finished by 16 March, and then turn to cleaning the code and final commenting while testing was finishing and traffic policies were compared during the week of 16 March.
\\


\begin{figure}[h]
\begin{center}
\includegraphics[scale=1.0]{gantt8Feb}
\caption{Gantt Chart from 8 Feb}
\end{center}
\end{figure}

\\

\begin{figure}[h]
\begin{center}
\includegraphics[scale=0.8]{gantt26Mar}
\caption{Gantt Chart from 26 March}
\end{center}
\end{figure}


\\
We completed tasks very much on schedule with what we planned at the beginning. In the interest of better code, we extended the Milestone 2 deadline to March 16; this allowed some desired behaviours to be completed. All coding was originally scheduled to stop 16 March to leave time for continued testing, commenting and code cleaning, and documentation. 

One member had issues with his code that put the team a week behind at the end. This was disappointing, but team members pulled together to complete the rest of the tasks on schedule and bring out our traffic simulation by the deadline.
\\

\subsection{Functional Requirements}

Final Characteristics 
- vehicles run on a map
- multiple maps in GIS standard
- variable number of vehicles
- vehicles make decisions to reach a goal
- vehicles exhibit passing behaviours 
- implement basic vehicle types: car, lorry
- cars have different behaviours: aggressive, patient
- vehicles appear visually different by type
- traffic flows bi-directionally
- junctions, signals, give way 
- multiple lanes on some/all roads
- speed limits on roads

Functional Requirements
- multiple maps in GIS standard
Vehicles run on map
Variable number of vehicles 


\subsection{Repast Simphony}

Overall we think our use of Repast Simphony was successful for our project. It provided a foundation API and Eclipse configuration that was not directly involved with agents or behaviours. It allowed us to create attractive visuals in the simulation and analysis graphing. We spent little time debugging, and were able to focus on simulating traffic. Simphony provided many extras (XML persistent storage, installers, etc) that we did not use due to lack of need or lack of implementation time.

An unforseen disadvantage of Simphony arose in that it cannot be installed or used on KCL lab computers. One member of our group did not have access to Simphony at home or on campus (the group did not know this until after the initial presentation), and he was not able to find a way to use Simphony until almost March, and then not effectively. This greatly reduced his ability to be useful to the team in coding and held up work on a major segment of the code.

Originally we considered using a belief-desire-intention (BDI) model to code the agents\textquoteright\ behaviour, but as the member intending to code this was unable to use Simphony, behaviours were instead coded into the agents themselves.

We originally intended to implement both signal and roundabout junctions, but the code for the signals and give way took much more work than anticipated, and so we decided not to implement roundabouts in this version of our code.
\\

\subsection{Further Work}

\textquotedblleft Validation data is usually macroscopic statistics such as flow rate, speeds and queue time, which can easily be compared with data from real traffic experiments. \textquotedblright [4d] 

\begin{itemize}\itemsep0pt
\item If we had more time we would have liked to download a road map from a town in England where we know road data/statistics and compare our model to the actual traffic flow for the same road. This would be a big step in validating our logic and making the code useful to others wishing to model traffic.

\item  Creating roundabouts was the next goal for our code, and this would make the code work for modelling most UK towns and cities. (Roundabouts are not generally used in the USA.)

\item  Further development can be done in displaying road closures for modelling construction or traffic accidents. This would make the model more realistic and useful in determining maintenance disruption.

\item Implement source/sinks for allowing traffic to come on/off the map in a more natural location than at junctions; perhaps from the edges of the map would be more optimal visually. This would allow traffic congestion to increase as more cars are fed into the simulation than are going out (and vice versa), and would allow modelling of road conditions created by end of workday, post-sport competition, etc. 

\item We did not make progress on the BDI as planned, so this was left out of the code. The code could be restructured to more neatly implement this if desired.

\item Adding other vehicle types such as motorbikes or road features such as pedestrian crossings or high occupancy vehicle (HOV) lanes to model their effect on congestion and traffic flow.
\end{itemize}


\section{PEER ASSESSMENT}


Waqar: 20

James: 18.5

Adeela: 20

Andrea: 20

Yoann: 21.5



\end{document}


